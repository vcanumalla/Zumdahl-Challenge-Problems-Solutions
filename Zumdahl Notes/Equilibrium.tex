\documentclass[9pt]{extarticle}
\usepackage{amsmath}
\usepackage{fancyhdr}
\usepackage[version =4]{mhchem}
\usepackage[paperheight = 3in, paperwidth = 5in, margin = 0.20in, top = 0.6in]{geometry}
\pagestyle{fancy}
\usepackage{pgf}
\usepackage{pgfpages}
\setlength{\parindent}{0cm}
\pgfpagesdeclarelayout{boxed}
{
  \edef\pgfpageoptionborder{0pt}
}
{
  \pgfpagesphysicalpageoptions
  {%
    logical pages=1,%
  }
  \pgfpageslogicalpageoptions{1}
  {
    border code=\pgfsetlinewidth{2pt}\pgfstroke,%
    border shrink=\pgfpageoptionborder,%
    resized width=\pgfphysicalwidth,%
    resized height=\pgfphysicalheight,%
    center=\pgfpoint{.5\pgfphysicalwidth}{.5\pgfphysicalheight}%
  }%
}

\pgfpagesuselayout{boxed}
\begin{document}



% \cardfrontstyle[\large\slshape]{headings}
% \cardbackstyle{empty}
\begin{center}
    \hspace{0pt}
    \vfill
    \Large Equilibrium \\
    \large Extent of Chemical Reactions
    \vfill
    \hspace{0pt}
\end{center}
% \cardfrontfoot{Atomic Theory}
\newpage

At equilibrium, the rate of the forward reaction is equal to the rate of the reverse reaction. For a reaction such as 
\begin{align*}
  \ce{H3PO4 + H2O <=> H2PO4- + H3O+}
\end{align*}
The amount of product being formed is equal to the amount of reactant formed. This is represented by the sign \ce{<=>}. No change is observed
because changes in one direction are balanced by changes in the other. The amount of product and reactant stay constant at this point.
\newpage
For any reaction in the form
\begin{align*}
  \ce{A <=> B}
\end{align*}
The rate law shows
\begin{align*}
  k_f[\ce{A}]^m = k_r[\ce{B}]^n
\end{align*}
We can also write the value for $K_{eq}$ to be 
\begin{align*}
\frac{k_f}{k_r} = \frac{[\ce{B}]^n}{[\ce{A}]^m} = K_{eq}
\end{align*} 

\end{document}