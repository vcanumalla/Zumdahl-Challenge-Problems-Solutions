\documentclass[11 pt]{article}
\usepackage[utf8]{inputenc}
\usepackage{amsmath,amssymb,amsthm}

\usepackage[pdftex]{graphicx}
\usepackage[shortlabels]{enumitem}
\usepackage[inline]{asymptote}
\usepackage{fancyhdr}
\usepackage{mathtools}
\usepackage[english]{babel}
\usepackage[margin = 1.5 in]{geometry}
\usepackage[nottoc]{tocbibind}
\usepackage{hyperref}
\usepackage{setspace}
\usepackage{csquotes}
\usepackage[final]{pdfpages}
\usepackage[version=4]{mhchem}
\usepackage{siunitx}
\usepackage[T1]{fontenc}
\usepackage{empheq}
\usepackage{import}
\usepackage{transparent}
\usepackage{xifthen}
\usepackage{mdframed}
\DeclareSIUnit\atm{atm}

\setlength{\tabcolsep}{12pt}
\newcommand{\icetable}[9]{
    \begin{tabular}{cc@{}c@{}c@{}c@{}c@{}c}

    & &   [\ce{#2#1}]& ${}+{}$ & [\ce{#4#3}] & \ce{<=>}& [\ce{#6#5}]\\
 
  I & &       #7    &&   #8                      &&  #9    \\
  C & &       $- #2x$    &&   $- #4x$                         &&  $+ #6x$    \\
  E & &       $(#7 - #2x)$    &&   $(#8 - #4x)$                         &&  $(#9 + #6x)$    \\
 
    \end{tabular}


}
\newcommand{\p}[1]{
  P_{\ce{#1}}
}
\newcommand{\m}[1]{
  M_{\ce{#1}}
}
\newmdenv[linewidth=1.5pt]{problemBox}
\newmdenv[linewidth = 1 pt]{answerBox}
\title{Chapter 6 Challenge Problem Solutions}
\author{Vishal Canumalla}

\begin{document}
    
    \maketitle
    
    \section*{6.97} 
    \begin{problemBox}
        \textbf{Problem:} Nitric oxide and bromine at initial pressures of 98.4 and 41.3 torr, respectively, were allowed to react at 300. K. At equilibrium the total pressure was 110.5 torr. The reaction is as follows.
        $$\ce{2NO + Br2 <=> 2NOBr}$$
        \begin{enumerate}[a)]
                \item Calculate the value of $K_p$
                \item What would be the partial pressures of all species if
            NO and \ce{Br2}, both at an initial partial pressure of
            0.30 atm, were allowed to come to equilibrium at
            this temperature?
        \end{enumerate}
    \end{problemBox}
    \textbf{Solution:} We model the change in pressure of each substance through an ICE table and note the partial pressures.
    \begin{center}
        \begin{tabular}{c|c@{}c@{}c@{}c@{}c}
        
          &   \ce{2NO(g)}      & $\ce{+}$ & $\ce{Br2(g)}$ & $\ce{<=>}$ & $2\ce{NOBr (g)}$  \\
      
      I   &        $98.4$       &            &   41.3               &     &                      \\
      C   &       $- 2x$        &            &   + $x$              &     &  + $2x$     \\
      E   &       $98.4 - x$    &            &   $41.3 - x$         &     &    $2x$     \\      
        \end{tabular}
    \end{center}
    We are told that $P_{E}$ = 110.5 torr, meaning that we can solve for $x$
    \begin{align*}
            P_E &= P_{\ce{NO}} + P_{\ce{Br2}} + P_{\ce{NOBr}} \\
                110.5 &= (98.4 - 2x) + (41.3 - x) + (2x) \\
                x &= 29.2 \text{ torr}
    \end{align*}
    Solving for each equilibrium concentration gives us a $K_p$ value of 
    \begin{answerBox}
          $$K_p = \frac{58.4^2}{40.0^2 \times 12.1} = 0.176 \text{ torr}^{-1} = 134 \text{ atm}^{-1}$$
    \end{answerBox}
    Solving for the second part of the problem becomes a bit more convoluted. We are given that the initial concentrations of the reactants \ce{NO} and \ce{Br2} are 
    0.3 atm each. Writing an ICE table for the following reaction is trivial and is omitted. We end with a $K_p$ of 
    $$134 =  \frac{(2x)^2}{(0.3-3x)^2(0.3-x)}$$
    Solving for $x$ is complicated to do algebraically, but graphically, we see that $x = 0.125$ atm. Knowing
    that the equilibrium concentrations of \ce{NO} and \ce{Br2} are $(0.3 - 2x)$ and $(0.3 - x)$, and 
    \ce{NOBr} is $2x$, we see that the partial pressures of each of the substances are
    \begin{answerBox}
      $$ P_{\ce{NO}} = \SI{0.052}{\atm} \qquad P_{\ce{Br2}} = \SI{0.18}{\atm} \qquad P_{\ce{NOBr}} = \SI{0.25}{\atm}$$
    \end{answerBox}
    \section*{6.99}
    \begin{problemBox}
    \textbf{Problem:} Consider the reaction $$\ce{P4(g) <=> 2P2(g)}$$ where $K_p = 1.00 \times 10^{-1}$ at \SI{1325}{\kelvin}.
    In an experiment where \ce{P4(g)} was placed in a container at 1325 K,  the equilibrium mixture of \ce{P4(g)} and \ce{P2(g)}
    has a total pressure of 1.00 atm. Calculate the equilibrium pressures of \ce{P4(g)} and \ce{P2(g)}. Calculate the fraction (by mole) of \ce{P4(g)}
    that has dissociated to reach equilibrium. \\
    \end{problemBox}
    \textbf{Solution:} In this problem, we have a few unknowns, including the original 
    pressure of \ce{P4(g)} and of course the equilibrium pressures of \ce{P4(g)} and \ce{P2(g)}.
    Our basic ICE table gives us two unknowns, the initial concentration of \ce{P4(g)} $y$,
    and the change variable $x$. I have omitted the ICE table again for this problem, but writing it out
    would give us the equilibrium concentrations of the two substances in terms of $y$ and $x$. 
    \begin{align*}
      \ce{P4(g)} &= y - x \\
      \ce{P2(g)} &= 2x 
    \end{align*}
    The problem statement also gives us the total pressure as $P = 1.00$ atm, which we can rewrite to get our system of equations
    \begin{align*}
      K_p &= \frac{(2x)^2}{(y-x)} \\
      P &= (y-x) + (2x)
    \end{align*}
    Solving this system of equations is simply algebraic manipulation, which gives us
    $y = \SI{0.865}{\atm}$, and $x = \SI{0.135}{\atm}$. Finding our answers is simply arithmetic.
    \begin{answerBox}
      $$P_{\ce{P4}} = \SI{0.73}{\atm} \qquad P_{\ce{P2}} = \SI{0.270}{\atm} \qquad 16\% \text{ of \ce{P4} decomposed}$$
    \end{answerBox}
    \section*{6.101}
    \begin{problemBox}
        \textbf{Problem:} Consider the reaction 
        $$\ce{3O2(g) <=> 2O3(g)}$$
        At $\SI{175}{\celsius}$ and a pressure of 128 torr an equilibrium mixture of \ce{O2} and \ce{O3} has a density of 0.168 g/L. Calculate  the $K_p$ 
        for the above reaction at \SI{175}{\celsius}. 
    \end{problemBox}
    \textbf{Solution:} We first must come up with a method of finding density in terms of pressure. Density's units are \si{\gram\per\liter},
    and the equation that relates pressure and volume is $$PV = nRT$$ However, we must manipulate this equation
    to introduce mass. We do this by introducing molar mass ($M$) and obtain $$PM = M\frac{n}{V}RT = \rho{RT}$$
    
     $$ \frac{PM}{RT} = \frac{P_{\ce{O2}}M_{\ce{O2}} + P_{\ce{O3}}M_{\ce{O3}}}{RT} = \rho = 0.168 $$
    
    We also know that $P = 128$ torr, or $\SI{.168}{\atm}$. Our second equation is $$P = P_{\ce{O2}} + P_{\ce{O3}} = 0.168$$
    This is a simple system of equations with two unknowns ($P_{\ce{O2}}$ and $P_{\ce{O3}}$). Once solved, we find that
    $P_{\ce{O2}} = \SI{0.118}{\atm}$ and $P_{\ce{O3}} = \SI{0.05}{\atm}$. Which gives us a final answer
    \begin{answerBox}
    $$K_p = \frac{{P_{\ce{O3}}}^2}{{P_{\ce{O2}}}^3} = \SI{1.5}{\per\atm}$$
    \end{answerBox}
    \newpage
    \section*{6.103}
    \begin{problemBox}
        \textbf{Problem:} A \SI{4.72}{\gram} sample of methanol (\ce{CH3OH}) was placed in an otherwise empty \SI{1.00}{\liter} flask and heated to
        \SI{250}{\celsius} to vaporize the methanol. Over time the methanol vapor decomposed by the following reaction:
        $$\ce{CH3OH(g) <=> CO(g) + 2H2(g)}$$ After the system has reached equilibrium, a tiny hole is drilled in the side of the flask
        allowing gaseous compounds to effuse out of the flask. Measurements of the effusing gas show that it contains
        33.0 times as much \ce{H2(g)} as \ce{CH3OH(g)}. Calculate the $K$ for this reaction at \SI{250}{\celsius}. 
    \end{problemBox}
    \textbf{Solution:} We are given that the effusing gas shows 33 times as much \ce{H2} as \ce{CH3OH}. However, this
    does not account for the the differing rate of effusion for the two gasses. Using Graham's law of effusion gets us an
    effusion ratio of
    $\frac{\text{Rate}_{\ce{H2}}}{\text{Rate}_{\ce{CH3OH}}} = \sqrt{\frac{M_{\ce{CH3OH}}}{M_{\ce{H2}}}} = \sqrt{\frac{32}{2}} = 4$. \\ \\
    We now know that \ce{H2} effuses at 4 times the rate of \ce{CH3OH}, however, we are told that there is 33 times as much \ce{H2}. Since the effusion
    rate does not account for enough of an increase, the other $\frac{33}{4} = 8.25$ factor comes from the greater concentration of \ce{H2} at equilibrium.
    The mole ratio of \ce{H2} to \ce{CH3OH} is thus $$\frac{n_{\ce{H2}}}{n_{\ce{CH3OH}}} = 8.25 $$\\ \\
    We start with 4.72 grams of \ce{CH3OH}, or \SI{0.147}{\mole}. The ICE Table is trivial and is omitted. Once written however,
    we see at equilibrium 
    \begin{align*}
      8.28 &= \frac{n_{\ce{H2}}}{n_{\ce{CH3OH}}} = \frac{2x}{0.147 - x} \\
      x &= 0.118
    \end{align*}
    Note that because the reaction takes place in a 1 liter vessel, the number of moles of any substance is identical
    in number to the concentration. 
    \begin{answerBox}
        $$K = \frac{4x^3}{0.147 - x} = 0.23$$
    \end{answerBox}
    \newpage
    \newpage
    \section*{6.105}
    \begin{problemBox}
        \textbf{Problem:} At \SI{207}{\celsius}, $K_p$ = \SI{0.267}{\atm} for the reaction
        $$\ce{PCl5(g) <=> PCl3(g) + Cl2(g)}$$
        \begin{enumerate}[a)]
            \item If \SI{0.100}{\mole} of \ce{PCl5(g)} is placed in an otherwise empty \SI{12.0}{\liter} vessel at \SI{207}{\celsius}, calculate the partial pressures of \ce{PCl5(g)}, \ce{PCl3(g)}, and \ce{Cl2(g)} at equilibrium.
            \item In another experiment the total pressure of an equilibrium mixture is \SI{2.00}{\atm{}} at \SI{207}{\celsius}. What mass of \ce{PCl5} was introduced into a \SI{5.00}{\liter} vessel to reach this equilibrium position?
        \end{enumerate} 
    \end{problemBox}
    \textbf{Solution:} We are given moles of \ce{PCl5}, when we need partial pressure to use it in $K_p$. We use the ideal gas equation to find partial pressure
    \begin{align*}
        PV &= nRT \\
        P(12) &= 0.1(0.0821)(480) \\
        P_{\ce{PCl5}} &= \SI{0.3284}{\atm{}}
    \end{align*}
    The ICE table is trivial and omitted, but gives us a $K_p$ expression of 
    $$K_p = \frac{x^2}{0.3284 - x} 0.267$$
    Solving this quadratic, we get $x = 0.191$, and the partial pressures at equilibrium are thus
    \begin{answerBox}
    \begin{align*}
        P_{\ce{PCl5}} = 0.1374 && P_{\ce{PCl3}} = 0.191 && P_{\ce{Cl2}} = 0.191
    \end{align*}
    \end{answerBox}
    For part b, we are asked to solve for the initial concentration of \ce{PCl5}. However, we are given neither
    the equilibrium concentration of \ce{PCl5} nor the value of the change variable $x$. It is evident
    that we will thus have to use a system of equations. Writing an ICE table and representing the initial
    concentration of \ce{PCl5} with the variable $y$ and the change variable as $x$, we obtain a $K_P$ expression of
    $$ K_P = \frac{P_{\ce{Cl2}} \cdot P_{\ce{PCl3}}}{P_{\ce{PCl5}}}$$
    We have two unknowns $x$ and $y$ and thus need a second equation to solve this system. This comes from
    the value for $P_{tot}$ at equilibrium, \SI{2.00}{\atm{}}. Our system of equations is thus
\begin{align*}
  0.267 &= \frac{x^2}{y-x} \\
  2.00 &= P_{\ce{PCl5}} + P_{\ce{PCl3}} + P_{\ce{Cl2}}= (y-x) + x + x 
\end{align*}
Solving this system of equations gets us an initial pressure of \SI{1.49}{\atm{}} for \ce{PCL5}, and converting to moles
using $PV = nRT$ gives us a mass of \boxed{\SI{39.4}{\gram} \ce{PCl5}} introduced. 
\section*{6.107}
\begin{problemBox}
  \textbf{Problem:} At 1000K the \ce{N2(g)} and \ce{O2(g)} in air (78\% \ce{N2}, 21\% \ce{O2}, by moles) react to form a mixture of \ce{NO(g)} and \ce{NO2(g)}.
  The values of the equilibrium constants are $1.5 \times 10^{-4}$ and $1.0 \times 10^{-5}$ atm$^{-1}$ for the formation of \ce{NO(g)} and \ce{NO2(g)}, respectively.
  At what total pressure will the partial pressures of \ce{NO(g)} and \ce{NO2(g)} be equal in an equilibrium mixture of \ce{N2(g)}, \ce{O2(g)}, \ce{NO(g)} and \ce{NO2(g)}?
\end{problemBox}
\textbf{Solution:} One particular ratio we can keep constant in this problem is the ratio between \ce{N2} and \ce{O2}, the ratio always being 78:21. We can call the equilibrium pressure of \ce{O2} to be $\p{O2}$, and the equilibrium pressure of \ce{N2} is thus 
 $$\p{N2} = (78/21)\p{O2} \approx 3.71\p{O2}$$
    \section*{6.109}
    \begin{problemBox}
      \textbf{Problem:} An \SI{8.00}{\gram} sample of \ce{SO3} was placed in an evacuated container, where it decomposed at \SI{600}{\celsius} according to the following reaction:
      $$\ce{SO3(g) <=> SO2(g) + \frac{1}{2}O2(g)}$$
      At equilibrium the total pressure and the density of the gaseous mixture were \SI{1.80}{\atm} and \SI{1.60}{\gram\per\liter}, respectively. Calculate $K_p$ for this reaction.
    \end{problemBox}
    \textbf{Solution:} This problem is rather similar to 6.101, with some extra complications. In order to solve, first we set up an ICE table.
    \begin{center}
      \begin{tabular}{c|c@{}c@{}c@{}c@{}c}
        
            &   \ce{SO3(g)}      & $\ce{<=>}$ & $\ce{SO2(g)}$ & $+$ & $\frac{1}{2}\ce{O2}$  \\
        
        I   &        $P_0$       &            &   0           &     &  0                    \\
        C   &       $- x$        &            &   + $x$       &     &  + $\frac{1}{2}x$     \\
        E   &       $P_0 - x$    &            &   $x$         &     &    $\frac{1}{2}x$     \\
        
      \end{tabular}
      
    \end{center}
  This gives us our base $K_p$ expression
  $$K_p = \frac{\p{SO2}{\p{O2}}^\frac{1}{2}}{\p{SO3}} = \frac{x \times (\frac{1}{2}x)^\frac{1}{2}}{\p{SO3} - x}$$
  and our substances' partial pressures at equilibrium
  \begin{align*}
    \p{SO3} &= \p{0} - x \\
    \p{SO2} &= x \\
    \p{O2} &= \frac{1}{2}x
  \end{align*}
  This $K_p$ expression relates the partial pressures of the 3 substances at equilibrium, meaning we can write the total pressure (1.8 atm) in terms of $x$ and $P_0$
  \begin{align*}
    1.80 &= \p{SO3} + \p{SO2} + \p{O2} \\
    1.80 &= (P_0 - x) + x + \frac{1}{2}x \\
    1.80 &= P_0 + \frac{1}{2}x
  \end{align*}
  Finally, we must derive an expression relating $x$, $\p{0}$ and density $\rho$. This is rather similar to the process done in 6.101, thus the reader is encouraged to review 6.101 if they have trouble deriving any steps.
  \begin{align*}
    \rho &= \frac{PM}{RT} \\
     \SI{1.6}{\gram\per\liter}   &= \frac{\p{SO3}\m{SO3} + \p{SO2}\m{SO2} + \p{O2}\m{O2}}{RT} \\
     &= \frac{(\p{0} - x)\m{SO3} + x\m{SO2} + \frac{1}{2}x\m{O2}}{RT} \\
     1.6 &= \frac{((1.80 - \frac{1}{2}x) - x)\m{SO3} + x\m{SO2} + \frac{1}{2}x\m{O2}}{RT}
  \end{align*}
  At this point, we can solve for $x$ through algebraic manipulation, and obtain a value of $x \approx  \SI{0.73}{\atm}$. This allows us to solve for $\p{0}$, and thereby the partial pressures of all substances. The final answer is thus 
  \begin{answerBox}
    $$K_p = 0.63$$ 
  \end{answerBox}
  
\end{document}