\documentclass[11 pt]{article}
\usepackage[utf8]{inputenc}
\usepackage{amsmath,amssymb,amsthm}

\usepackage[pdftex]{graphicx}
\usepackage[shortlabels]{enumitem}
\usepackage[inline]{asymptote}
\usepackage{fancyhdr}
\usepackage{mathtools}
\usepackage[english]{babel}
\usepackage[margin = 1.5 in]{geometry}
\usepackage[nottoc]{tocbibind}
\usepackage{hyperref}
\usepackage{setspace}
\usepackage{csquotes}
\usepackage[final]{pdfpages}
\usepackage[version=4]{mhchem}
\usepackage{siunitx}
\usepackage[T1]{fontenc}
\usepackage{empheq}
\usepackage{import}
\usepackage{transparent}
\usepackage{xifthen}
\usepackage{mdframed}

\setlength{\tabcolsep}{12pt}
\newcommand{\icetable}[9]{
    \begin{tabular}{cc@{}c@{}c@{}c@{}c@{}c}

    & &   [\ce{#2#1}]& ${}+{}$ & [\ce{#4#3}] & \ce{<=>}& [\ce{#6#5}]\\
 
  I & &       #7    &&   #8                      &&  #9    \\
  C & &       $- #2x$    &&   $- #4x$                         &&  $+ #6x$    \\
  E & &       $(#7 - #2x)$    &&   $(#8 - #4x)$                         &&  $(#9 + #6x)$    \\
 
\end{tabular}

}
\title{Chapter 6 Challenge Problem Solutions}
\author{Vishal Canumalla}

\begin{document}

\maketitle

\section*{6.97} 

\textbf{Problem:} Nitric oxide and bromine at initial pressures of 98.4 and 41.3 torr, respectively, were allowed to react at 300. K. At equilibrium the total pressure was 110.5 torr. The reaction is as follows.
$$\ce{2NO + Br2 <=> 2NOBr}$$
\begin{enumerate}[a)]
    \item Calculate the value of $K_p$
    \item What would be the partial pressures of all species if
NO and \ce{Br2}, both at an initial partial pressure of
0.30 atm, were allowed to come to equilibrium at
this temperature?
\end{enumerate}
\textbf{Solution:} We model the change in pressure of each substance through an ICE table and note the concentrations.
\begin{center}
 \icetable{NO}{2}{Br2}{}{NOBr}{2}{98.4}{41.3}{}
\end{center}
We are told that $P_{E}$ = 110.5 torr, meaning that we can solve for $x$
\begin{align*}
P_E &= P_{\ce{NO}} + P_{\ce{Br2}} + P_{\ce{NOBr}} \\
    110.5 &= (98.4 - 2x) + (41.3 - x) + (2x) \\
    x &= 29.2 \text{ torr}
\end{align*}
Solving for each equilibrium concentration gives us a $K_p$ value of 
$$K_p = \frac{58.4^2}{40.0^2 \times 12.1} = 0.176 \text{ torr} = \boxed{137 \text{ atm}^{-1}}$$
Solving for the second part of the problem becomes a bit more convoluted. We are given that the initial concentrations of the reactants \ce{NO} and \ce{Br2} are 
0.3 atm each. Writing an ICE table for the following reaction is trivial and is omitted. We end with a $K_p$ of 
$$134 =  \frac{(2x)^2}{(0.3-3x)^2(0.3-x)}$$
Solving for x is complicated to do algebraically, but graphically, we see that $x = 0.125$ atm. Knowing
that the equilibrium concentrations of \ce{NO} and \ce{Br2} are $(0.3 - 2x)$ and $(0.3 - x)$, and 
\ce{NOBr} is $2x$, we see that the partial pressures of each of the substances are
\begin{align*}
    P_{\ce{NO}} &= 0.052 \text{atm} \\
    P_{\ce{Br2}} &= 0.18 \text{atm} \\
    P_{\ce{NOBr}} &= 0.25 \text{atm}  
\end{align*}
\section*{6.99}
\textbf{Problem:} Consider the reaction $$\ce{P4(g) <=> 2P2(g)}$$ where $K_p = 1.00 \times 10^{-1}$ at \SI{1325}{\kelvin}.
In an experiment where \ce{P4(g)} was placed in a container at 1325 K,  the equilibrium mixture of \ce{P4(g)} and \ce{P2(g)}
has a total pressure of 1.00 atm. Calculate the equilibrium pressures of \ce{P4(g)} and \ce{P2(g)}. Calculate the fraction (by mole) of \ce{P4(g)}
that has dissociated to reach equilibrium. \\

\textbf{Solution:} In this problem, we have a few unknowns, including the original 
pressure of \ce{P4(g)} and of course the equilibrium pressures of \ce{P4(g)} and \ce{P2(g)}.
Our basic ICE table gives us two unknowns, the initial concentration of \ce{P4(g)} $y$, a
and the change variable $x$. I have omitted the ICE table again for this problem, but writing it out
would give us the equilibrium concentrations of the two substances in terms of $y$ and $x$. 
\begin{align*}
  \ce{P4(g)} &= y - x \\
  \ce{P2(g)} &= 2x 
\end{align*}
The problem statement also gives us the total pressure as $P = 1.00$ atm, which we can rewrite to get our system of equations
\begin{align}
  K_p &= \frac{(2x)^2}{(y-x)} \\
  P &= (y-x) + (2x)
\end{align}
Solving this system of equations is simply algebraic manipulation, which gives us
$y = 0.865$ atm, and $x = 0.135$ atm. Finding our answers is simply arithmetic.

\end{document}